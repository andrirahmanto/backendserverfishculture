%!TEX root = ./template-skripsi.tex
%-------------------------------------------------------------------------------
%                            	BAB IV
%               		KESIMPULAN DAN SARAN
%-------------------------------------------------------------------------------

\chapter{KESIMPULAN DAN SARAN}

\section{Kesimpulan}
Berdasarkan hasil implementasi dan pengujian fitur sistem web service yang telah dirancang, maka diperoleh kesimpulan sebagai berikut:

\begin{enumerate}
	\item Terimplementasikannya prototipe backend web service aqua breeding versi pertama, yang berfokus kepada fitur pencatatan dan visualisasi data budidaya ikan. Adapun perancangannya dilakukan dengan metode Scrum dengan tahapan penyusunan Product Backlog, Sprint Backlog, dan dikerjakan dalam sepuluh Sprint.
	
	\item Berdasarkan hasil pengujian web service yang dilakukan terhadap frontend developer, didapatkan bahwa API berhasil diimplementasikan kedalam prototipe aplikasi aqua breeding berbasis android. 
	
	\item Berdasarkan hasil User Acceptance Test terhadap satu pembudidaya, didapatkan bahwa prototipe API dan tampilan admin versi pertama sudah sesuai dan menghasilkan kesimpulan untuk fitur yang akan di tambahkan pada versi kedua.
\end{enumerate}

\section{Saran}
Adapun saran untuk penelitian selanjutnya adalah:
\begin{enumerate} 
	\item Berdasarkan diskusi dengan owner, harus dimulainya pengembangan frontend sistem dengan mengintegrasikan API yang sudah di buat ke aplikasi aqua breeding berbasis android agar sistem dapat dipakai oleh para pembudidaya. 
	\item Berdasarkan diskusi dengan owner, pada versi selanjutnya menambahkan fitur multi farm, multi user, login, dan logout secara API maupun aplikasi. Agar dapat dirilis dan dapat dipakai secara umum oleh pembudidaya lain. 
\end{enumerate}


% Baris ini digunakan untuk membantu dalam melakukan sitasi
% Karena diapit dengan comment, maka baris ini akan diabaikan
% oleh compiler LaTeX.
\begin{comment}
\bibliography{daftar-pustaka}
\end{comment}