\chapter*{\centering{\large{ABSTRAK}}}

\begin{spacing}{1}
\textbf{ANDRI RAHMANTO}. Perancangan Arsitektur Aplikasi Budidaya Perikanan Modern Pada \textit{Backend} yang Bertanggung Jawab Dalam Melayani Transaksi \textit{Query Webservice}. Skripsi. Fakultas Matematika dan Ilmu Pengetahuan Alam, Universitas Negeri Jakarta. 2023. Di bawah bimbingan Muhammad Eka Suryana, M.Kom dan Med Irzal, M.Kom.
\newline
\newline
Budidaya ikan air tawar merupakan salah satu sumber perikanan negara Indonesia. Pencatatan indikator seperti jumlah ikan, suhu air, ph air, pemberikan pakan wajib dilakukan dalam pembudidayaan ikan air tawar, baik untuk menentukan tindakan terhadap keadaan kolam maupun menentukan harga jual. Pada umumnya pencatatan indikator - indikator pada budidaya ikan dilakukan secara konvensional yang rawan akan kesalahan dalam perhitungan. Penelitian ini bertujuan untuk membuat arsitektur \textit{backend web service API} untuk aplikasi budidaya perikanan modern. Jenis Penelitian ini adalah Pengembangan/\textit{Research} and Development. Data diambil dari hasil diskusi bersama pembudidaya ikan air tawar JFT \textit{(J Farm Technology)} dan studi literatur dengan membaca jurnal-jurnal yang terkait dengan topik penelitian. Diskusi menghasilkan suatu \textit{user requirement} yang menjadi acuan fitur yang akan diterapkan pada arsitektur \textit{backend web service API}. Proses pengembangan sistem ini menggunakan metode \textit{Scrum} yang dilakukan dalam 10 \textit{sprint} dan seluruh program yang dibuat menggunakan bahasa pemrograman \textit{Python}. Sistem diuji dengan melakukan 2 pengujian yaitu \textit{Unit Test} dan \textit{UAT} .Hasil akhir dari arsitektur ini mencakup struktur \textit{project}, rancangan diagram, \textit{web service} berupa \textit{REST API}, serta dokumentasi lengkap \textit{API}.
\newline
\newline
\noindent \textbf{Kata kunci}: \textit{web service, REST API, budidaya ikan, perikanan modern, backend, scrum}
\end{spacing}