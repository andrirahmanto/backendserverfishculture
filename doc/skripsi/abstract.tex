\chapter*{\centering{\large{ABSTRACT}}}

\begin{spacing}{1}
\textbf{ANDRI RAHMANTO}. Designing the Architecture of Modern Aquaculture Application in the Responsible Backend for Serving Webservice Query Transactions. Undergraduate Thesis. Faculty of Mathematics and Natural Sciences, Universitas Negeri Jakarta. 2023. Supervised by Muhammad Eka Suryana, M.Kom and Med Irzal, M.Kom.
\newline
\newline
Freshwater fish farming is one of Indonesia's significant fisheries resources. Recording indicators such as fish quantity, water temperature, water pH, and feeding is essential in freshwater fish cultivation, both to determine pond conditions and establish selling prices. Generally, the recording of these indicators in fish farming is done conventionally, which is susceptible to calculation errors. This research aims to create a backend web service API architecture for modern aquaculture applications. The research type is Development/Research and Development. Data is gathered through discussions with freshwater fish farmers from JFT (J Farm Technology) and a literature study involving relevant research articles. These discussions result in user requirements that serve as a reference for the features to be implemented in the backend web service API architecture. The system development process employs the Scrum methodology, conducted over 10 sprints, and all programs are developed using the Python programming language. The system is tested through two tests: Unit Testing and User Acceptance Testing (UAT). The final outcome of this architecture includes the project structure, diagram designs, a REST API-based web service, and comprehensive API documentation.
\newline
\newline
\noindent \textbf{Keywords}: web service, REST API, fish farming, modern aquaculture, backend, scrum
\end{spacing}